 \documentclass[a4paper,11pt]{article}

\usepackage[utf8]{inputenc}
\usepackage[italian]{babel}
\usepackage{graphicx}
\usepackage{float} 
\usepackage[colorlinks=true,linkcolor=blue]{hyperref}
\usepackage{nameref} 

\graphicspath{{../Configurazioni/}}
\def\code#1{\texttt{#1}}
\def\image[#1][#2]#3{
  \begin{figure}[H]
  \centering
  \includegraphics[#2]{#1}
  \caption{#3}
  \end{figure}}
\def\boximage[#1][#2]#3{
  \begin{figure}[H]
  \centering
  \fbox{\includegraphics[#2]{#1}}
  \caption{#3}
  \end{figure}}
 
 
 

\title{Progettazione e configurazione di una rete aziendale}
\author{Filippo Mariani, Giorgio Mazza}


\begin{document}

\maketitle
\newpage
\tableofcontents
\newpage

\section{Requisiti del progetto e finalità}
Un'azienda di onoranze funebri richiede la progettazione e configurazione di una rete aziendale in grado di interconnettere i vari edifici di suddetta azienda.
Gli edifici sono i seguenti:
\begin{table}[H]
\centering
\label{riepilogo}
\begin{tabular}{|l|l|l|}
\hline
\multicolumn{1}{|c|}{\textbf{Nome edificio}} & \multicolumn{1}{c|}{\textbf{Numero utenti}} \\ \hline
Agenzia & 50 \\ \hline
Barificio & 260 \\ \hline
Cremino & 200 \\ \hline
Destinazione & 50 \\ \hline
Epitaffio & 30 \\ \hline
\end{tabular}
\caption{Edifici}
\end{table}
ed è richiesto il seguente schema di collegamenti:
\image[../Configurazioni/SchemaFisico.png][scale=0.3]{Schema Fisico}
\newpage
\paragraph{Sono richiesti:} 
\begin{itemize}
\item accesso protetto a Internet
\item copertura WiFi nell'edificio C
\item un server di posta elettronica
\item un server di backup 
\item un server adibito alle applicazioni aziendali (da proteggere con particolare attenzione)
\end{itemize}
In oltre, in base a quanto richiesto, si è deciso di configurare:
\begin{itemize}
\item tre server DNS
\item un server proxy 
\item un server DHCP
\end{itemize}
I vari server forniscono servizi agli utenti della rete e servono per il funzionamento stesso della rete.\\
Il \textbf{Server Mail} si occupa di fornire servizi per la posta elettronica, il \textbf{Server Web} permette a tutte le postazioni della rete di poter accedere alla rete Internet.\\
Nel \textbf{Server per le applicazioni aziendali} risiedono programmi applicativi in uso all’Azienda.\\
I tre \textbf{Server DNS} si occupano di tradurre un indirizzo della forma http://www.mazziamari.it, in uno espresso in forma numerica (richiesta per l’accesso ad Internet). Per questo motivo tutte le macchine che accedono ad Internet devono avere specificato il loro DNS Server.\\
Nel \textbf{Server di Backup} verranno salvati i dati generati dai vari programmi applicativi in uso in modo da garantire una copia di sicurezza.\\
I sopracitati server sono così disposti:
\begin{table}[H]
\centering
\label{riepilogo}
\begin{tabular}{|l|l|l|}
\hline
\multicolumn{1}{|c|}{\textbf{Nome edificio}} & \multicolumn{1}{c|}{\textbf{Server}} \\ \hline
Agenzia & - \\ \hline
Barificio & DNS, Mail, Web, Proxy \\ \hline
Cremino & DNS, DHCP, App. aziendali \\ \hline
Destinazione & - \\ \hline
Epitaffio & DNS, Backup \\ \hline
\end{tabular}
\caption{Dislocazione Server}
\end{table}
\newpage

\section{Schema Logico della rete}
La topologia della rete diverge dallo schema fisico della stessa: i quattro edifici più vicini - Agenzia, Barificio, Cremino e Destinazione) sono collegati tramite fibra ottica, così da garantire un servizio veloce, efficente ed affidabile. 
\\In quanto all'edificio Epitaffio, dato che è molto distante dagli altri, si è deciso di connetterlo ai restanti tramite Virtual Private Network (\textbf{VPN}):\\ ciò permette di costruire una rete privata virtuale su un'infrastruttura pubblica, usando una rete non dedicata (in questo caso Internet).
L'uso di VPN per l'edificio EPitaffio, rende la soluzione di gran lunga più economica, semplice e sicura rispetto all'uso di altre tecnologie come la fibra ottica.
\image[../Configurazioni/SchemaLogico.png][scale=0.12]{Schema Logico}
\paragraph{Struttura della rete}
Alla rete principale è stato assegnato l'IP di classe B \textbf{192.168.0.0/24}, al quale corrisponde il nome di dominio \\ \textbf{http://www.mazziamari.it}; Questa è stata suddivisa in sottoreti, ognuna corrispondente a ciascun edificio:
\begin{itemize}
\item Agenzia: 192.168.1.0/24
\item Barificio: 192.168.2.0/24 e 192.168.22.0/24
\item Cremino: 192.168.3.0/24
\item Destinazione: 192.168.4.0/24
\item Epitaffio: 192.168.5.0/24
\end{itemize}

\paragraph{DMZ}
Per il collegamento a internet degli edifici Barificio ed Epitaffio, si è deciso di introdurre una De-Militarized Zone (DMZ), posta per l'appunto agli estremi della rete, a svolgere funzioni di sicurezza e per contenere i server che forniscono servizi accessibili anche dall'esterno della rete. Questa rete sarà trattata in dettaglio successivamente nei singoli edifici che la contengono.
\newpage
\subsection{A: Agenzia}
\image[../Configurazioni/Agenzia/Agenzia.png][scale=0.28]{Dettaglio sottoreterete 1}
All'edificio Agenzia è stata assegnata la sottorete con indirizzo \textbf{192.168.1.0/24} (unica sottorete per l'edificio, visto il numero esiguo di utenti che deve ospitare) \\
Questa è raggiungibile tramite il router \textbf{Anubi} con indirizzo \textbf{192.168.1.3}, tramite le interfacce \textit{192.168.2.12} per \textit{Barificio} e \textit{192.168.2.12} per \textit{Cremino}.
\subsection{B: Barificio}
\image[../Configurazioni/Barificio/Barificio.png][scale=0.25]{Dettaglio sottoreterete 2}

\subsection{C: Cremino}
\image[../Configurazioni/Cremino/Cremino.png][scale=0.28]{Dettaglio sottoreterete 3}

\subsection{D: Destinazione}
\image[../Configurazioni/Destinazione/Destinazione.png][scale=0.28]{Dettaglio sottoreterete 4}

\subsection{E: Epitaffio}
\image[../Configurazioni/Epitaffio/Epitaffio.png][scale=0.25]{Dettaglio sottoreterete 5}





\end{document}