 \documentclass[a4paper,11pt]{article}

\usepackage[utf8]{inputenc}
\usepackage[italian]{babel}
\usepackage{graphicx}
\usepackage{float} 
\usepackage[colorlinks=true,linkcolor=blue]{hyperref}
\usepackage{nameref} 

\graphicspath{{../Configurazioni/}}
\def\code#1{\texttt{#1}}
\def\image[#1][#2]#3{
  \begin{figure}[H]
  \centering
  \includegraphics[#2]{#1}
  \caption{#3}
  \end{figure}}
\def\boximage[#1][#2]#3{
  \begin{figure}[H]
  \centering
  \fbox{\includegraphics[#2]{#1}}
  \caption{#3}
  \end{figure}}
 
 
 

\title{Progettazione e configurazione di una rete aziendale}
\author{Filippo Mariani, Giorgio Mazza}


\begin{document}

\maketitle
\newpage
\tableofcontents
\newpage

\section{Requisiti del progetto e finalità}
Un'azienda di onoranze funebri richiede la progettazione e configurazione di una rete aziendale in grado di interconnettere i vari edifici di suddetta azienda.
Gli edifici sono i seguenti:
\begin{table}[H]
\centering
\label{riepilogo}
\begin{tabular}{|l|l|l|}
\hline
\multicolumn{1}{|c|}{\textbf{Nome edificio}} & \multicolumn{1}{c|}{\textbf{Numero utenti}} \\ \hline
Agenzia & 50 \\ \hline
Barificio & 260 \\ \hline
Cremino & 200 \\ \hline
Destinazione & 50 \\ \hline
Epitaffio & 30 \\ \hline
\end{tabular}
\caption{Edifici}
\end{table}
ed è richiesto il seguente schema di collegamenti:
\image[../Configurazioni/SchemaFisico.png][scale=0.3]{Schema Fisico}
\newpage
\paragraph{Sono richiesti:} 
\begin{itemize}
\item accesso protetto a Internet
\item copertura WiFi nell'edificio C
\item un server di posta elettronica
\item un server di backup 
\item un server adibito alle applicazioni aziendali (da proteggere con particolare attenzione)
\end{itemize}
In oltre, in base a quanto richiesto, si è deciso di configurare:
\begin{itemize}
\item tre server DNS
\item un server proxy 
\item un server DHCP
\end{itemize}
I vari server forniscono servizi agli utenti della rete e servono per il funzionamento stesso della rete.\\
Il \textbf{Server Mail} si occupa di fornire servizi per la posta elettronica, il \textbf{Server Web} permette a tutte le postazioni della rete di poter accedere alla rete Internet.\\
Nel \textbf{Server per le applicazioni aziendali} risiedono programmi applicativi in uso all’Azienda.\\
I tre \textbf{Server DNS} si occupano di tradurre un indirizzo della forma \url{http://www.mazziamari.it}, in uno espresso in forma numerica (richiesta per l’accesso ad Internet). Per questo motivo tutte le macchine che accedono ad Internet devono avere specificato il loro DNS Server.\\
Nel \textbf{Server di Backup} verranno salvati i dati generati dai vari programmi applicativi in uso in modo da garantire una copia di sicurezza.\\
I sopracitati server sono così disposti:
\begin{table}[H]
\centering
\label{riepilogo}
\begin{tabular}{|l|l|l|}
\hline
\multicolumn{1}{|c|}{\textbf{Nome edificio}} & \multicolumn{1}{c|}{\textbf{Server}} \\ \hline
Agenzia & - \\ \hline
Barificio & DNS, Mail, Web, Proxy \\ \hline
Cremino & DNS, DHCP, App. aziendali \\ \hline
Destinazione & - \\ \hline
Epitaffio & DNS, Backup \\ \hline
\end{tabular}
\caption{Dislocazione Server}
\end{table}
\newpage

\section{Schema Logico della rete}
La topologia della rete diverge dallo schema fisico della stessa: i quattro edifici più vicini - Agenzia, Barificio, Cremino e Destinazione) sono collegati tramite fibra ottica, così da garantire un servizio veloce, efficente ed affidabile. 
\\In quanto all'edificio Epitaffio, dato che è molto distante dagli altri, si è deciso di connetterlo ai restanti tramite Virtual Private Network (\textbf{VPN}):\\ ciò permette di costruire una rete privata virtuale su un'infrastruttura pubblica, usando una rete non dedicata (in questo caso Internet).
L'uso di VPN per l'edificio EPitaffio, rende la soluzione di gran lunga più economica, semplice e sicura rispetto all'uso di altre tecnologie come la fibra ottica.
\image[../Configurazioni/SchemaLogico.png][scale=0.12]{Schema Logico}
\paragraph{Struttura della rete}
Alla rete principale è stato assegnato l'IP di classe B \textbf{192.168.0.0/24}, al quale corrisponde il nome di dominio \\ \textbf{http://www.mazziamari.it}; Questa è stata suddivisa in sottoreti, ognuna corrispondente a ciascun edificio:
\begin{itemize}
\item Agenzia: 192.168.1.0/24
\item Barificio: 192.168.2.0/24 e 192.168.22.0/24
\item Cremino: 192.168.3.0/24
\item Destinazione: 192.168.4.0/24
\item Epitaffio: 192.168.5.0/24
\end{itemize}

\paragraph{DMZ}
Per il collegamento a internet degli edifici Barificio ed Epitaffio, si è deciso di introdurre una De-Militarized Zone (DMZ), posta per l'appunto agli estremi della rete, a svolgere funzioni di sicurezza e per contenere i server che forniscono servizi accessibili anche dall'esterno della rete. Questa rete sarà trattata in dettaglio successivamente nei singoli edifici che la contengono.
\newpage
\subsection{A: Agenzia}
\image[../Configurazioni/Agenzia/Agenzia.png][scale=0.28]{Dettaglio sottoreterete 1}
All'edificio Agenzia è stata assegnata la sottorete con indirizzo \textbf{192.168.1.0/24} (unica sottorete per l'edificio, visto il numero esiguo di utenti che deve ospitare) \\
Questa è raggiungibile tramite il router \textbf{Anubi} con indirizzo \textbf{192.168.1.3}, tramite le interfacce \textit{192.168.2.12} per \textit{Barificio} e \textit{192.168.3.13} per \textit{Cremino}.

\begin{table}[H]
\centering
\label{riepilogo}
\begin{tabular}{|l|l|l|}
\hline
\multicolumn{1}{|c|}{\textbf{Dispositivo}} & \multicolumn{1}{c|}{\textbf{Nome}} & \multicolumn{1}{c|}{\textbf{IP}} \\ \hline
Router & Anubi & 192.168.1.3 \\ \hline
Host 1 & agenzia01 & 192.168.1.50 \\ \hline
... & ... & ...\\ \hline
Host 50 & agenzia50 & 192.168.1.100 \\ \hline
\end{tabular}
\caption{Riepilogo A}
\end{table}



\subsection{B: Barificio}
\image[../Configurazioni/Barificio/Barificio.png][scale=0.25]{Dettaglio sottoreterete 2}

All'edificio Barificio sono state assegnate le sottoreti con indirizzo \textbf{192.168.2.0/24} e \textbf{192.168.22.0/24} (due sottorete per l'edificio B, visto il numero cospicuo di utenti che deve ospitare) \\
Queste sono raggiungibili tramite il router \textbf{Basilisco} con indirizzo 212.10.2.3, tramite le interfacce \textit{192.168.1.21} per \textit{Agenzia} e \textit{192.168.3.23} per \textit{Cremino}, che comunica cone le due sottoreti mediante le sue interfacce \textbf{192.168.2.3} e \textbf{192.168.22.3}.

\begin{table}[H]
\centering
\label{riepilogo}
\begin{tabular}{|l|l|l|}
\hline
\multicolumn{1}{|c|}{\textbf{Dispositivo}} & \multicolumn{1}{c|}{\textbf{Nome}} & \multicolumn{1}{c|}{\textbf{IP}} \\ \hline
Interfaccia1 & Basilisco & 192.168.2.3 \\ \hline
Host 1 & barificio01 & 192.168.2.50 \\ \hline
... & ... & ...\\ \hline
Host 50 & barificio40 & 192.168.2.90 \\ \hline
Interfaccia2 & Basilisco & 192.168.22.3 \\ \hline
Host 51 & barificio41 & 192.168.22.30 \\ \hline
... & ... & ...\\ \hline
Host 260 & barificio260 & 192.168.22.260 \\ \hline
\end{tabular}
\caption{Riepilogo B}
\end{table}


\subsubsection{DMZ}
Nell’edificio B è presente anche la DMZ, nella quale sono inclusi i server che devono risultare accessibili anche dall’esterno della rete. Alla DMZ è stata assegnata la rete \textbf{212.10.2.0/24}.\\
Agli estremi della DMZ sono presenti \textbf{due router / firewall} che svolgono funzione di filtraggio e protezione per la rete: 
\begin{itemize}
\item il firewall-router interno \textbf{Basilisco} con indirizzo \textbf{212.10.2.3}  protegge la DMZ dagli accessi provenienti dalla LAN
\item il firewall-router esterno \textbf{Barista}, che comunica con internet, con indirizzo \textbf{212.10.2.6} protegge la DMZ dagli accessi esterni.
\end{itemize}
In particolare il firewall collocato tra la LAN e la DMZ filtra i pacchetti in transito mediante le due catene \textit{dmzlan e landmz}, svolgendo anche la funzione di NAT.\\
Allo stesso modo il firewall esterno usa le due catene \textit{inetdmz e dmzinet} per effettuare \textit{packet filtering} tra Internet e la DMZ.


\begin{table}[H]
\centering
\label{riepilogo}
\begin{tabular}{|l|l|l|}
\hline
\multicolumn{1}{|c|}{\textbf{Dispositivo}} & \multicolumn{1}{c|}{\textbf{Nome}} & \multicolumn{1}{c|}{\textbf{IP}} \\ \hline
Firewall/Router Interno & Basilisco & 212.10.2.3 \\ \hline
Firewall/Router Esterno & Barista & 212.10.2.6 \\ \hline
DNS server & Bouquet & 212.10.2.10 \\ \hline
Mail server & Mogano & 212.10.2.20 \\ \hline
Web server & Noce & 212.10.2.30 \\ \hline
Proxy server & Rovere & 212.10.2.40 \\ \hline
\end{tabular}
\caption{Riepilogo B}
\end{table}


\subsection{C: Cremino}
\image[../Configurazioni/Cremino/Cremino.png][scale=0.28]{Dettaglio sottoreterete 3}
All'edificio Cremino è stata assegnata la sottorete con indirizzo \textbf{192.168.3.0/24}\\
Questa è raggiungibile tramite il router \textbf{Caronte} con indirizzo \textbf{192.168.3.3}, tramite le interfacce \textit{192.168.1.31} per \textit{Agenzia}, \textit{192.168.2.32} per \textit{Barificio} e \textit{192.168.4.34} per \textit{Destinazione}.\\
La sottorete è usufruibile dagli host in forma wireless, mediante gli access points collocati in vari punti dell'edificio; E’ una rete che può presentare un numero variabile di hosts (max 200) in base a quanti utenti vi si connettono. Per questo motivo, si è deciso di assegnare agli hosts presenti un \textbf{indirizzo dinamico} tramite DHCP in un range che varia da 192.168.3.50 a .250.\\
Di conseguenza, nel suddetto edificio sarà presente un \textbf{server DHCP} (e un server per le applicazioni aziendali).\\
Infine è presente un server DNS interno, adibito alla risoluzione dei nomi degli host della LAN.


\begin{table}[H]
\centering
\label{riepilogo}
\begin{tabular}{|l|l|l|}
\hline
\multicolumn{1}{|c|}{\textbf{Dispositivo}} & \multicolumn{1}{c|}{\textbf{Nome}} & \multicolumn{1}{c|}{\textbf{IP}} \\ \hline
Router & Caronte & 192.168.3.3 \\ \hline
DNS server & Camino & 192.168.3.10 \\ \hline
DHCP server & Corteo & 192.168.3.20 \\ \hline
Server app aziendali & Social & 192.168.3.30 \\ \hline
Host 1 & agenzia01 & 192.168.3.50 \\ \hline
... & ... & ...\\ \hline
Host 50 & agenzia50 & 192.168.3.250 \\ \hline
\end{tabular}
\caption{Riepilogo C}
\end{table}

\newpage
\subsection{D: Destinazione}
\image[../Configurazioni/Destinazione/Destinazione.png][scale=0.28]{Dettaglio sottoreterete 4}
All'edificio Destinazione è stata assegnata la sottorete con indirizzo \textbf{192.168.4.0/24} (unica sottorete per l'edificio, visto il numero esiguo di utenti che deve ospitare) \\
Questa è raggiungibile tramite il router \textbf{Dite} con indirizzo \textbf{192.168.3.3}, tramite l'interfaccia \textit{192.168.3.43} per \textit{Cremino}.

\begin{table}[H]
\centering
\label{riepilogo}
\begin{tabular}{|l|l|l|}
\hline
\multicolumn{1}{|c|}{\textbf{Dispositivo}} & \multicolumn{1}{c|}{\textbf{Nome}} & \multicolumn{1}{c|}{\textbf{IP}} \\ \hline
Router & Dite & 192.168.3.3 \\ \hline
Host 1 & destinazione01 & 192.168.3.50 \\ \hline
... & ... & ...\\ \hline
Host 50 & destinazione50 & 192.168.3.100 \\ \hline
\end{tabular}
\caption{Riepilogo D}
\end{table}


\subsection{E: Epitaffio}
\image[../Configurazioni/Epitaffio/Epitaffio.png][scale=0.25]{Dettaglio sottoreterete 5}
All'edificio Epitaffio è stata assegnata la sottorete con indirizzo \textbf{192.168.5.0/24} (unica sottorete per l'edificio, visto il numero esiguo di utenti che deve ospitare) 

\begin{table}[H]
\centering
\label{riepilogo}
\begin{tabular}{|l|l|l|}
\hline
\multicolumn{1}{|c|}{\textbf{Dispositivo}} & \multicolumn{1}{c|}{\textbf{Nome}} & \multicolumn{1}{c|}{\textbf{IP}} \\ \hline
Interfaccia & Ecate & 192.168.5.3 \\ \hline
Host 1 & epitaffio01 & 192.168.5.30 \\ \hline
... & ... & ...\\ \hline
Host 30 & epitaffio40 & 192.168.5.60 \\ \hline
Backup server & Memorie & 192.168.5.10 \\ \hline
\end{tabular}
\caption{Riepilogo E}
\end{table}

\subsubsection{Rete intermedia} 
\image[../Configurazioni/DMZ.png][scale=0.2]{Dettaglio rete intermedia delle DMZ}
L'edificio E deve poter essere raggiunto sia dall'edificio B che dall'edificio D: data la distanza da essi, si è deciso di farli comunicare mediante \textbf{VPN}.\\
A tal scopo abbiamo deciso di far comunicare E direttamente con internet (con opportuna aggiunta di DMZ) ed abbiamo creato una rete privata che connette la DMZ di E con la DMZ di B, corrispondente all'indirizzo \textbf{212.11.0.0/16}. Ciò è reso possibile facendo comunicare l'interfaccia \textit{212.11.5.2} del router esterno di E \textbf{Epigrafia} con indirizzo \textbf{212.10.5.6}, con l'interfaccia \textit{212.11.2.5} del router esterno di B.\\
Così facendo, E può comunicare con B mediante suddetta rete e con D passando per B e C.


\subsubsection{DMZ}
Nell’edificio E è presente anche la DMZ, nella quale è presente il server DNS. Alla DMZ è stata assegnata la rete \textbf{212.10.5.0/24}.\\
Agli estremi della DMZ sono presenti \textbf{due router / firewall} che svolgono funzione di filtraggio e protezione per la rete: 
\begin{itemize}
\item il firewall-router interno \textbf{Ecate} con indirizzo \textbf{212.10.5.3}  protegge la DMZ dagli accessi provenienti dalla LAN
\item il firewall-router esterno \textbf{Epigrafia}, che comunica con internet, con indirizzo \textbf{212.10.5.6} protegge la DMZ dagli accessi esterni.
\end{itemize}
I firewall lavorano in modo analogo a quelli della DMZ dell'edificio B.


\begin{table}[H]
\centering
\label{riepilogo}
\begin{tabular}{|l|l|l|}
\hline
\multicolumn{1}{|c|}{\textbf{Dispositivo}} & \multicolumn{1}{c|}{\textbf{Nome}} & \multicolumn{1}{c|}{\textbf{IP}} \\ \hline
Firewall/Router Interno & Ecate & 212.10.5.3 \\ \hline
Firewall/Router Esterno & Epigrafia & 212.10.5.6 \\ \hline
DNS server & Eredità & 212.10.5.10 \\ \hline
\end{tabular}
\caption{Riepilogo DMZ E}
\end{table}
\newpage

\section{Routing}
Per quanto riguarda il routing all'interno della rete:
\begin{itemize}
\item Per gli edifici A, B, C si è deciso di utilizzare un routing dinamico configurando il protocollo \textbf{RIP} (Routing Information Protocol), un protocollo di routing interno basato su metrica vettore-distanza, gestito nel nostro caso dal demone \textit{gated}.
\item Per gli edifici E, D e la rete intermedia 212.11.0.0 si è deciso di utilizzare un routing statico tramite il comando \textbf{route add}, essendoci una sola strada percorribile per giungervi.
\end{itemize}

\section{DNS}
Nella rete sono presenti tre server DNS: due esterni, che risiedono nelle due DMZ, ed uno interno.\\
\begin{itemize}
\item I DNS esterni sono situati nella DMZ dell'edificio B (\textit{Bouquet}) e in quella dell'edificio E (\textit{Eredità}) e sono 
\begin{itemize}
\item master dei server della DMZ (i due DNS traducono in nomi gli indirizzi dei server cui si accede dall'esterno - ciò permette la presenza online dell'azienda)
\item slave tra di loro per la risoluzione dei nomi di internet
\end{itemize} 
\item Il DNS interno, posto nell'edificio C (\textit{Camino}), è:
\begin{itemize}
\item master delle reti interne (esso permette di accedere più comodamente alle varie macchine in rete)
\item slave dei due DNS esterni (chiede ai precedenti per quanto riguarda i server della DMZ e per la risoluzione dei nomi di internet)
\end{itemize} 
\end{itemize}
Ciò rende anche più sicura l'intera rete in quanto dall'esterno non è possibile accedere ai nomi interni.
\newpage
\section{Sicurezza}
\subsection{Firewalls}
Per soddisfare gli standard minimi di sicurezza, si è scelto di adoperare due firewall distinti per DMZ (quattro in totale), configurati tramite \code{iptables}, integrati rispettivamente nei due router Barificio e nei due router di Epitaffio. \\Grazie ad essi, la DMZ risulta protetta sia dagli accessi provenienti dalla rete locale che da quelli esterni, come sopra spiegato.
\subsection{Hardening: il server per applicazioni aziendali}
La sicurezza del server per applicazioni aziendali è stata rafforzata filtrando i pacchetti TCP con un wrapper e facendovi girare il superserver \code{xinetd}, che a sua volta monitora le richieste ai servizi telnet, SSH e NFS talvolta, come nel caso di telnet, disabilitandoli completamente.

\newpage
\section{Preventivo di spesa}

\begin{table}[H]
	\centering
	\label{costo componentistica}
	\resizebox{\textwidth}{!}{\begin{tabular}{ll|l|l|}
			\hline
			\multicolumn{1}{|c|}{\textbf{Componente}} & \multicolumn{1}{c|}{\textbf{Quantità}} & \multicolumn{1}{c|}{\textbf{Prezzo cad.}} & \textbf{Prezzo tot.} \\ \hline
			\multicolumn{1}{|l|}{TP-Link TL-SF1048 Switch 48 Porte} & 4 & 154.34 EUR & 926.04 EUR \\ \hline
			\multicolumn{1}{|l|}{WiFi access point} & 3 & 130 EUR & 390 EUR \\ \hline
			\multicolumn{1}{|l|}{Fibra ottica} & 2100 m & 6.34 EUR/m & 13314 EUR \\ \hline
			\multicolumn{1}{|l|}{Cavo UTP} & 3000 m & 2.65 EUR/m & 7950 EUR \\ \hline
			\multicolumn{1}{|l|}{VPN} &  & 300 EUR annui & 300 EUR annui \\ \hline
			&  & TOTALE: & \textbf{22880.04 EUR} \\ \cline{3-4} 
	\end{tabular}}
	\caption{Costo delle componenti hardware}
\end{table}
\ \\Al costo della componentistica va sommato il costo dell'installazione, pari a 20.000 euro, per un totale complessivo di 42880.04 euro.

\end{document}